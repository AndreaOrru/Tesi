% Preambolo
\documentclass[12pt,a4paper]{report}
\usepackage[utf8]{inputenc}
\usepackage[T1]{fontenc}
\usepackage[english,italian]{babel}
\usepackage[norules]{frontespizio}

% Documento
\begin{document}
\frenchspacing

% Frontespizio
\begin{frontespizio}
	\Istituzione{Università di Pisa}
	\Logo[6cm]{logo}
	\Dipartimento{Informatica}
	\Corso[Laurea triennale]{Informatica}
	\Annoaccademico{2013--2014}
	\Titoletto{Tesi di laurea}
	\Titolo{Implementazione di un sistema operativo \\ UNIX-like basato su microkernel}
	\Candidato[452264]{Andrea Orrù}
	\Relatore{Antonio Cisternino}
	\Rientro{1.5cm}
\end{frontespizio}

% Abstract
\renewcommand{\abstractname}{Abstract}
\begin{abstract}
	In questa tesi presenteremo \emph{Utopia}, un sistema operativo \emph{UNIX-like} basato su \emph{micro-kernel}.
	Partiremo da una presentazione generale dei sistemi operativi, descrivendo le motivazioni
	alla base della loro esistenza, e le funzionalità che essi forniscono solitamente.
	Vedremo poi come questo insieme di funzioni possa essere realizzato in vari modi.
	Infine discuteremo le scelte implementative relative al caso particolare di Utopia.
\end{abstract}

% Indice
\tableofcontents


% Introduzione
\chapter*{Introduzione}
\addcontentsline{toc}{chapter}{Introduzione}
	Questa è un'introduzione.


% Capitoli
\chapter{Kernel}
	Questo è un capitolo sui kernel.
	\section{Microkernel}
		Questa è una sezione sui microkernel.


% Bibliografia
\begin{thebibliography}{}
	\bibitem{Silberschatz}
		Silberschatz A., Galvin P. B., Gagne G.,
		\emph{Operating System Concepts}.
		Wiley,
		9th Edition,
		2012.
\end{thebibliography}

\end{document}
